\documentclass[a4paper,11pt]{article}
\usepackage[utf8]{inputenc}
\usepackage[spanish]{babel}
\usepackage{csquotes} 
\selectlanguage{spanish}

\usepackage[backend=biber, style=ieee]{biblatex}
\addbibresource{referencias.bib}

\usepackage[a4paper,tmargin=3cm,bmargin=2.5cm,lmargin=3cm, rmargin=2.5cm, bindingoffset=6mm]{geometry}

\usepackage{enumitem}
\usepackage{amsmath, amssymb, amsfonts}
\usepackage{mathrsfs}
\usepackage{amsthm}

%======Insertat Script======
\usepackage{algpseudocode}

%======Insertar Color Texto======
\usepackage[usenames]{color}

%======Insertar Color Especial======
\usepackage [dvipsnames] { xcolor }

%======Insertar Hipervínculo======
\usepackage{hyperref}
\hypersetup{
	colorlinks=true,
	linkcolor=black,
	filecolor=red,      
	urlcolor=blue,
}

%========================================================================= 
\title{}
\author{Michael Paucar}
\date{12 de abril de 2020}

\begin{document}
\maketitle

\section*{Teorema Maestro}\noindent
{\color {Gray} \rule {\linewidth}{0.1mm}}\newline%Insertar Lineas
En el análisis de algoritmos, el \textbf{teorema maestro} proporciona una soluciòn sencilla en tèrminos asintóticos (usando una \href{https://es.wikipedia.org/wiki/Cota_superior_asintótica}{Cota superior asintótica}) para \href{https://es.wikipedia.org/w/index.php?title=Ecuaciones_de_recurrencia&action=edit&redlink=1}{\textcolor{Maroon}{ecuaciones de recurrencia}} que ocurren en muchos algoritmos recursivos tales como en el
\href{https://es.wikipedia.org/wiki/Algoritmo_divide_y_vencerás}{Algoritmo divide y vencerás}. Fue popularizado por el libro \href{https://es.wikipedia.org/wiki/Introducción_a_los_algoritmos}{\textit{Introducción a los algoritmos}} por \href{https://es.wikipedia.org/wiki/Thomas_H._Cormen}{Cormen}, \href{https://es.wikipedia.org/wiki/Charles_E._Leiserson}{Leiserson}, \href{https://es.wikipedia.org/wiki/Ron_Rivest}{Rivest}, y \href{https://es.wikipedia.org/wiki/Clifford_Stein}{Stein} en el cual fue tanto introducido como probado formalmente. No todas las ecuaciones de recurrencia pueden ser resueltas con el uso del teorema maestro.\cite{wiki:xxx}

\addtocontents{toc}{\color{Gray}\protect\hypertarget{toc}{}} 
\addtocontents{toc}{\vspace{-2mm} \hspace{-7.5mm} \hrule \par}
	\tableofcontents
\noindent{\color {Gray} \rule {\linewidth}{0.1mm}}	

\newpage
\section[Introducción]{\hyperlink{toc}{Introducción}}
\noindent{\color {Gray} \rule {\linewidth}{0.1mm}}
Considere un problema que puede ser resuleto  a través de un \href{https://es.wikipedia.org/wiki/Algoritmo_recursivo}{algoritmo recursivo} como el siguiente:\\

\begin{algorithmic}
\State \textbf{procedimiento} T( n : tamaño del problema ) \textbf{definición}:
\If{ $n < 1$} \textbf{exit}\\
\State Hacer trabajo f(n)\\
\State T$(n/b)$
\State T$(n/b)$
\State ...repetir una cantidad de \textit{a} veces...
\State T$(n/b)$
\State \textbf{fin procedimiento}
\EndIf\\
\end{algorithmic}

\noindent
En el \href{https://es.wikipedia.org/wiki/Algoritmo_recursivo}{algoritmo recursivo} mostrado arriba se divide el problema original en una cantidad menor de manera recursiva, cada uno de estos subproblemas siendo del tamaño de \textit{n$/$b}. Esto podría seer visualizado como la construcción de un \href{https://es.wikipedia.org/w/index.php?title=árbol_de_llamadas&action=edit&redlink=1}{\textcolor{Maroon}{árbol de llamadas}} al problema, siendo cada nodo del árbol una instancia recursiva del mismo y sus hijos siendo instancias de las llamadas subsiguientes, En el emplo de arriba, cada nodo tendría un número $a$ de hijos. Cada \href{https://es.wikipedia.org/wiki/Nodo}{nodo} hará la cantidad de trabajo correspondiente al tamaño del subproblema \textit{n}, pasado a esa instancia en particular, estando la cantidad de trabajo realizado determinado por \textbf{$f(n)$}. El trabajo total realizado por todas las llamadas del árbol es la suma de los trabajos hechos por cada uno de los nodos del árbol.\\
Algoritmos como el decrito arriba pueden ser representados por una \href{https://es.wikipedia.org/wiki/Relación_de_recurrencia}{relación de recurrencia} como la siguiente: $T(n) = a\;T (\dfrac{n}{b}) + f(n)$. Esta relación de recurrencia puede ser sucesivamente sustituida en sí misma y expandida hasta obtener una expresión que reprresente la cantidad total de trabajo.\cite{Duke_University}\\
El teorema maestro nos permite fácilmente calcular el tiempo de ejecución de este tipo de algoritmos recursivos en notación de Cota superior asintótica, sin la necesidad de hacer la expansión de las distintas llamadas recursivas.


\section[Forma genérica]{\hyperlink{toc}{Forma genérica}}
\noindent{\color {Gray} \rule {\linewidth}{0.1mm}}
El teorema maestro sirve para resolver relaciones recursivas de la siguiente forma:\\
$T(n) = a\; T (\dfrac{n}{b}) + f(n) \;donde\; a \geq 1,b > 1$\\ 
En la aplicación de análisis de algoritmos recursivos, las constantes y funciones toman los siguientes significados:

\begin{itemize}
    \item $n$ es el tamaño del problema a resolver
    \item $a$ es el número de subproblemas en la recursión
    \item $n/b$ el tamaño de cada subproblema. (Aquí se asume que todos los subproblemas tienen el mismo tamaño)
    \item $f(n)$ es el costo del trabajo hecho fuera de las llamadas recursivas, que incluye el costo de la división del problema y el costo de unir las soluciones de los subproblemas.
\end{itemize}

Luego, es posible realizar una cota en notación Big O en los siguientes tres casos:

\subsection{Caso 1}
\subsubsection{Forma Genérica} \noindent
Si $f(n) = O \;(n^{c})$ donde $c < \log_b a$ (usando \href{https://es.wikipedia.org/wiki/Cota_superior_asintótica}{Cota superior asintótica})\\ entonces:\\
$T(n) = \Theta (n^{\log_b a})$
\subsubsection{Ejemplo}
$T(n) = 8T (\dfrac{n}{2}) + 1000n^{2}$\\
Como puede verse en la fórmula de arriba:\\
$a=8$, $b=2$, $f(n)=1000n^{2}$; entonces\\
$f(n) = O (n^{c})$, donde $c = 2$\\
Luego, vemos que se cumple a condición del caso 1:\\
$\log_b a = \; \log_2 8 = 3 > c$\\
Luego por el teorema maestro tenemos que\\
$T(n) = \Theta (n^{log_b a})=\Theta (n^{3})$

\subsection{Caso 2}
\subsubsection{Forma Genérica}\noindent
Si es verdad que para alguna constante $k \geq 0$, que:\\
$f(n) = \Theta (n^{c} \log^{k} n)$ donde $c=\log_b a$\\
Entonces:\\
$T(n)=\Theta(n^{c} \log^{k+1} n)$
\subsubsection{Ejemplo}
$T(n)=2T(\dfrac{n}{2})+10\;n$\\
Como podemos ver en la fórmula de arriba, las variables tienen los siguientevalores:\\
$a=2$\\$b=2$\\$c=1$\\$f(n)=10n$\\
$f(n)=\Theta(n^{c}\log^{k}n)$ donde $c=1$, $k=0$\\
Luego, nos fijamos que cumpla la condición del caso.\\
$\log_b a = \log_2 2=1$, luego, se cumple que $c=\log_b a$\\
Entonces por el segundo caso del teorema maestro:\\
$T(n)=\Theta(n^{\log_b a}\log^{k+1} n)=\Theta=(n^{1} \log^{1} n)=\Theta=(n\log n)$\\
Dando de esa manera que la relación de recurrencia de $T(n)$ es $\Theta$ es $\Theta (n \log n)$.

\subsection{Caso 3}
\subsubsection{Forma Genérica}\noindent
Si es verdad además que:\\$f(n)=\omega(n^{c})$ donde $c>\log_b a$\\
Y si es verdad además que:\\
$af(\dfrac{n}{b})\leq kf(n)$ para alguna constante $k<1$ y suficientemente grande $n$\\ Entonces:\\
$T(n)=\Theta(f(n))$
\subsubsection{Ejemplo}
$T(n)=2T (\dfrac{n}{2})+n^{2}$\\$f(n)=\omega (n^{c})$ donde $c=2$\\
Luego se comprueba que satisface la condición del caso 3.\\
$\log_b a = \log_2 2 = 1$, y por lo tanto se cumple que, $c>\log_ba$\\
La condición también se cumple:\\ $2 (\dfrac{n^{2}}{4}) \leq kn^{2}$, eligiendo $k=1/2$\\
Entonces por el tercer caso del teorema maestro:\\$T(n)=\Theta(f(n))=\Theta n^{2}$,\\Por consiguiente la relación de recurrencia $T(n)$ es in ($\Theta (n^{2})$)

\section[Casos Irresolubles]{\hyperlink{toc}{Casos Irresolubles}}
\noindent
{\color {Gray} \rule {\linewidth}{0.1mm}}
Los siguientes casos no pueden ser resuletos a través de la utilización del teorema maestro:\cite{Massachusetts}

\begin{itemize}
    \item $T(n)=2^{n}\;T(\dfrac{n}{2})+n^{n}$\\
    $a$ no es una constante, el número de subproblemas debe ser fijo.
    \item $T(n)=2T(\dfrac{n}{2})+\dfrac{n}{\log n}$\\
    $f(n)$ debe ser polinomial.
    \item $T(n)=0.5 T (\dfrac{n}{2})+n$\\
    $a < 1$ no puede darse el caso de que haya menos de un subproblema.
    \item $T(n)=64T(\dfrac{n}{8})-n^{2} \log n$\\
    $f(n)$ que es el tiempo de trabajo, no puede ser negativo.
    \item $T(n)=T(\dfrac{n}{2})+n(2-\cos n)$\\
    Caso 3 pero hay una violación de regularidad.
\end{itemize}

\newpage
\section[Bibliografía]{\hyperlink{toc}{Bibliografía}}
{\color {Gray} \rule {\linewidth}{0.1mm}}
\begin{itemize}[label={$\bullet$}]
	\item \href{https://es.wikipedia.org/wiki/Thomas_H._Cormen}{Thomas H. Cormen}, \href{https://es.wikipedia.org/wiki/Charles_E._Leiserson}{Charles E. Leiserson}, \href{https://es.wikipedia.org/wiki/Ronald_L._Rivest}{Ronald L. Rivest}, and \href{https://es.wikipedia.org/wiki/Clifford_Stein}{Clifford Stein}. \href{https://es.wikipedia.org/wiki/Introduction_to_Algorithms}{Introduction to Algorithms}, Second Edition. \href{https://es.wikipedia.org/wiki/MIT_Press}{MIT Press} and McGraw-Hill, 2001. \href{https://es.wikipedia.org/wiki/Especial:FuentesDeLibros/0262032937}{ISBN 0-262-03293-7}. Sections 4.3 (The master method) and 4.4 (Proof of the master theorem), pp. 73–90.
	
	\item \href{https://es.wikipedia.org/w/index.php?title=Michael_T._Goodrich&action=edit&redlink=1}{\textcolor{Maroon}{Michael T. Goodrich}} and \href{https://es.wikipedia.org/w/index.php?title=Roberto_Tamassia&action=edit&redlink=1}{\textcolor{Maroon}{Roberto Tamassia}}. Algorithm Design: Foundation, Analysis, and Internet Examples. Wiley, 2002. \href{https://es.wikipedia.org/wiki/Especial:FuentesDeLibros/0471383651}{ISBN 0-471-38365-1}. The master theorem (including the version of Case 2 included here, which is stronger than the one from CLRS) is on pp. 268-270.
\end{itemize}

\section[Veáse también]{\hyperlink{toc}{Veáse también}}
{\color {Gray} \rule {\linewidth}{0.1mm}}
\begin{itemize}[label={$\bullet$}]
    \item \href{https://es.wikipedia.org/wiki/Ecuación_recurrente}{Ecuación recurrente}
\end{itemize}
	
\section[Referencias]{\hyperlink{toc}{Referencias}}
{\color {Gray} \rule {\linewidth}{0.1mm}}
%NO INSERTA  DOBLE TÌTULO
\defbibheading{bibliography}[\refname]{} 
\printbibliography
\end{document}
